\documentclass{article}
\usepackage[utf8]{inputenc}
\usepackage{amsmath}
\title{Tunnelling Times in Quantum Mechanics}
\author{James Puleston}

\begin{document}

\maketitle

\section{Larmor Precession}

We consider the case of scattering in one dimension with particles of mass $m$, spin $\frac{1}{2}$ and kinetic energy $
E = \frac{\hbar^2k^2}{2m}$. The particles move along the y-axis with spins aligned with the x-axis and interact with a rectangular barrier,

\begin{equation}
	V = 
	\begin{cases}
	V_0 & -\frac{d}{2}<y<\frac{d}{2}\\
		0 & \text{otherwise}
	\end{cases}
\end{equation}

A small magnetic field $\vec{B_{0}}$ points along the z-axis and is confined to the barrier. As particles enter the barrier, the magnetic field induces a Larmor precession with frequency $\omega_{L}=\frac{g \mu B_{0}}{\hbar}$, where $g$ is the gyromagnetic ratio, $\mu$ is the absolute value of the magnetic moment. 

\end{document}
