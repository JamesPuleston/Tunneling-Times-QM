\documentclass{article}
\usepackage[utf8]{inputenc}
\usepackage[margin=1in]{geometry}
\usepackage{amsmath}
\usepackage{braket}
\usepackage{bbold}
\title{Tunnelling Times in Quantum Mechanics}
\author{James Puleston}

\begin{document}

\maketitle

\section{Larmor Precession}

We consider the case of scattering in one dimension with particles of mass $m$, spin $\frac{1}{2}$ and kinetic energy $
E = \frac{\hbar^2k^2}{2m}$. The particles move along the y-axis with spins polarised with the x-axis and interact with a rectangular barrier,

\begin{equation}
	V = 
	\begin{cases}
	V_0 & -\frac{d}{2}<y<\frac{d}{2}\\
		0 & \text{otherwise}
	\end{cases}
\end{equation}

A small magnetic field $\vec{B_{0}}$ points along the z-axis and is confined to the barrier. As particles enter the barrier, the magnetic field induces a Larmor precession with frequency $\omega_{L}=\frac{g \mu B_{0}}{\hbar}$, where $g$ is the gyromagnetic ratio, $\mu$ is the absolute value of the magnetic moment. The polarisations of the transmitted and reflected particles are compared with the polarisation of the incident particles.

Particles initially polarised in the x direction obtain a y and z components when tunnelling through the barrier. We know from the Stern-Gerlach experiment that particles polarised int eh x direction can be represented as combinations of particles with z polarisations, $\ket{x; \pm} = \frac{1}{\sqrt{2}}\ket{z;+} \pm \frac{1}{\sqrt{2}}\ket{z;-}$. Outside the barrier, particles have kinetic energy $E$, independent of their spin. Inside the barrier, the kinetic energy differs by the Zeeman contribution $\pm \frac{\hbar \omega_{L}}{2}$. The wavefunction inside the barrier will contain an exponentially decaying term $Exp(\kappa_{\pm})$, where $\kappa_{\pm} = (k^{2}_{0}-k^{2} \pm \frac{m \omega_L}{\hbar})^{\frac{1}{2}}$, where $\kappa = (k_{0}^2-k^2)^{\frac{1}{2}}$ and the sign indicates spin parallel or antiparallel to the field.  

We can approximate this in the small $\omega_L$ limit as
	\begin{align*}
		\kappa_{\pm} &= \left(k^{2}_{0}-k^{2} \mp \frac{m \omega_{L}}{\hbar}\right)^{\frac{1}{2}}\\	
			     &= \kappa \left(1 \mp \frac{m \omega_{L}}{\hbar \kappa^{2}}\right)^{\frac{1}{2}}\\
			     &\approx \kappa \left(1 \mp \frac{m \omega_{L}}{2\hbar \kappa^{2}}\right)\\
			      &= \kappa \mp \frac{m \omega_{L}}{2\hbar \kappa}
	\end{align*}

Here we examine tunnelling through a barrier in a magnetic field. In this case our Hamiltonian is
\begin{equation}
	H = 
	\begin{cases}
	\left(\frac{p^2}{2m} + V_0\right)\mathbb{1}-\left(\frac{\hbar \omega_L}{2}\right) \sigma_z & |y| \leq \frac{d}{2}\\
	\left(\frac{p^2}{2m}\right)\mathbb{1} & |y| \geq \frac{d}{2}
	\end{cases}
	\end{equation}
where $\mathbb{1}$ is the $2 \times 2$ identity matrix and $\sigma_{x}, \sigma_{y}, \sigma_{z}$ are the Pauli spin matrices.

H acts on spinors
\begin{align}
	\psi &= \begin{pmatrix}
		\psi_{+}(y) \\
		\psi_{-}(y)
		\end{pmatrix}
\end{align}

As usual $|\psi_{\pm}(y)|^{2}dy$ is the probability of finding a particle \textit{upon measurement} with spin $\pm \frac{\hbar}{2}$ in the interval $y, y+dy$. We emphasise the 'upon measurement' here as this is an important point of distinction between the orthodox and pilot-wave interpretations addressed in this essay. The incident beam is polarised in the x direction,
\begin{align}
	\psi &= \frac{1}{\sqrt{2}}
	\begin{pmatrix}
	1\\
	1
	\end{pmatrix}
	e^{i k y}
\end{align}
i.e. $\psi$ is an eigenvector of $S_{x}$

H is diagonal in the spinor basis so we can solve the scattering problem for particles with spin $\frac{\hbar}{2}$ and $-\frac{\hbar}{2}$ separately. 
Our wavefunction is of the form
\begin{equation}
	\psi = 
	\begin{cases}
		A_{\pm}e^{i k y} + B_{\pm}e^{-i k y} & y \leq -\frac{d}{2} \\
		C_{\pm}e^{\kappa_{\pm}y} + D_{\pm}e^{-\kappa_{\pm}y} & -\frac{d}{2} \leq y \leq \frac{d}{2} \\
		F_{\pm}e^{i k y} & y \geq \frac{d}{2}
	\end{cases}
\end{equation}

We will soon set $A_{\pm} = 1$, corresponding to 1 particle per ??, but maintain it for now to aid a future calculation. Note there is no $e^{-i k y}$ term on the right of the barrier, as no particles are reflected. 

The effect of the magnetic field $B_0$ is simply to adjust the height of the barrier, $V_0^{'} = V_0 \pm \frac{\hbar \omega_L}{2}$. Hence we can solve the scattering problem initially assuming no magnetic field, and then adjusting our solution by replacing $\kappa$ in the field-free problem with $\kappa_{\pm}$. Our job now is to calculate the wavefunction coefficients A, B, C, D, F using the continuity of the wavefunction and its first derivative at the boundaries.
This is a lengthy calculation but the results are used so frequently that it is necessary to include a derivation. The results are stated here and derived below:

\begin{align}
	F &= T^{\frac{1}{2}}e^{i\Delta\phi}e^{-ikd} & B &= R^{\frac{1}{2}}e^{-\frac{i\pi}{2}}e^{i\Delta\phi}e^{-ikd} \nonumber \\
	C &= \frac{\kappa+ik}{2\kappa}e^{\frac{ikd}{2}}e^{\frac{-\kappa d}{2}}F & D &= \frac{\kappa-ik}{2\kappa}e^{\frac{ikd}{2}}e^{\frac{\kappa d}{2}}F \label{cont0}
\end{align}
where $T$ is t:he transmission probability and $R = 1-T$ is the reflection probability.

First we introduce a new coordinate system so that the boundaries of our barrier become 0, d. Then, denoting our wavefunctions before, inside and after the barrier as $\psi_{1}, \psi_{2}, \psi_{3}$ respectively, we have:

\begin{align}
	\psi_{1} &= Ae^{iky} + Be^{-iky} & \psi_{1}^{'} &= Aike^{iky} - ikBe^{-iky} \\
	\psi_{2} &= Ce^{-\kappa y} + De^{\kappa y} & \psi_{2}^{'} &= -\kappa Ce^{-\kappa y} + \kappa De^{\kappa y} \\
	\psi_{3} &= Fe^{iky} & \psi_{3}^{'} &= ikFe^{iky}
\end{align}

Imposing continuity of the wavefunction and its first derivative at the barrier boundaries:
\begin{align}
	\psi_{1}(0) = \psi_{2}(0) &\implies A+B = C+D \label{cont1}\\
	\psi_{1}^{'}(0) = \psi_{2}^{'}(0) &\implies ikA - ikB = -\kappa C + \kappa D \label{cont2}\\
	\psi_{2}(d) = \psi_{3}(d) &\implies Ce^{-\kappa d} + De^{\kappa d} = Fe^{ikd} \label{cont3}\\
	\psi_{2}^{'}(d) = \psi_{3}^{'}(d) &\implies -\kappa Ce^{-\kappa d} + \kappa De^{\kappa d} = ikF e^{ikd} \label{cont4}
\end{align}

\begin{align}
	ik(\ref{cont1})+(\ref{cont2}) &\implies 2ikA = C(ik-\kappa)+D(ik+\kappa) \label{cont5}\\
	ik(\ref{cont1})-(\ref{cont2}) &\implies 2ikB = C(ik+\kappa)+D(ik-\kappa) \label{cont6}\\
	\kappa(\ref{cont3})-(\ref{cont4}) &\implies 2\kappa Ce^{\kappa d} = Fe^{ikd}(\kappa-ik) \label{cont7}\\
	\kappa(\ref{cont3})+(\ref{cont4}) &\implies 2\kappa De^{\kappa d} = Fe^{ikd}(\kappa+ik) \label{cont8}
\end{align}

Inserting equations (\ref{cont7}) and (\ref{cont8}) into equation (\ref{cont5}) we arrive at:

\begin{align}
	2ikA &= -\frac{(ik-\kappa)^2}{2\kappa}Fe^{(ik+\kappa)d}+\frac{(ik+\kappa)^2}{2p}Fe^{(ik-\kappa)d} \label{cont9} \\
	\implies 4\kappa ikAe^{-ikd} &= F[(k^2-\kappa^2)(e^{\kappa d}-e^{-\kappa d})+2ik\kappa(e^{\kappa d}+e^{-\kappa d})] \label{cont10}\\
				     &= F[2(k^2-\kappa^2)\sinh{\kappa d}+4ik\kappa \cosh{\kappa d}] \label{cont11}
\end{align}

We hence arrive at our first result, the transmission probability $T = \frac{|F|^2}{|A|^2}$:
\[
	T = [1+\frac{(k^2+\kappa^2)^2\sinh^2{\kappa d}}{4k^2\kappa^2}]^{-1}
.\] 

It is clear now why we left the coefficient A explicit, but from now on it will be set to A = 1.

Rearranging (\ref{cont11}) for $F$, we can compare with the final result in (\ref{cont0}) and factorise out $T^\frac{1}{2}$:

\[
	F = T^{\frac{1}{2}}e^{-ikd} \times \frac{(k^2-\kappa^2)i\sinh{\kappa d}+2k\kappa \cosh(\kappa d)}{\sqrt{(k^2+\kappa^2)^2\sinh^{2}{\kappa d}+4k^2\kappa^2}}
\] 
This yields the identification

\begin{align}
	e^{i\Delta\phi} &= \frac{(k^2-\kappa^2)i\sinh{\kappa d}+2k\kappa \cosh{\kappa d}}{\sqrt{(k^2+\kappa^2)\sinh^{2}(\kappa d)+4k^2\kappa^2}} \\
			&= \frac{(k^2-\kappa^2)i\tanh{\kappa d}+2k\kappa}{\sqrt{(k^2-\kappa^2)^2\tanh^2{\kappa d}+4k^2\kappa^2}}
\end{align}

Expanding the left hand side into real and imaginary parts and comparing coefficients we deduce:

\[
	\tan{\Delta\phi} = \frac{(k^2-\kappa^2)\tanh(\kappa d)}{2k\kappa}
\] 

The result for B follows along similar lines and results for C and D follow immediately from equations (\ref{cont7}) and (\ref{cont8}).


\end{document}
